\section{Wissenschaft}
\authors{Annika Dunkel und Josua Kugler}
\subsection{Was ist Wissenschaft?}
Die Wissenschaft kann in drei Hauptbereiche untergliedert werden:
 Gesellschaftswissenschaften,  Naturwissenschaften und Geisteswissenschaften.

\noindent Ziel der Wissenschaft sind Erkenntnis und Fortschritt, welche durch empirische Experimente o.\,ä. und Schlussfolgerungen vorangebracht werden. Wichtig für wissenschaftliche Thesen bzw. Theorien ist die Falsifizierbarkeit, da nicht falsifizierbare Thesen nicht zur Erkenntnis beitragen. Das liegt daran, dass der Wahrheitsgehalt solcher Thesen nicht bestimmt werden kann und sie damit für Wissenschaft, die auf Logik beruht, keinen Nutzen haben.
Die Wissenschaft ist also ein zusammenhängendes System von Aussagen, Theorien und Verfahrensweisen.
Dieses System wird ständig auf dessen Wahrheitsgehalt hin überprüft und ist mit dem Anspruch objektiver, überpersönlicher Gültigkeiten verbunden.

\noindent Der Erfolg der Wissenschaft hängt von der richtigen Kommunikation ab. Beispielsweise wurde 2012 ein möglicher Beweis für eine wichtige mathematische Vermutung, die $abc$-Vermutung, veröffentlicht, bisher konnte dieser Beweis oder die zugehörige Theorie allerdings nicht nachvollzogen werden.\\
\subsection{Was ist Mathematik?}
Es ist schwierig, den Begriff der Mathematik zu definieren, da jeder etwas anderes darunter versteht. Für manche zählen Hypothesen zur Mathematik, für wieder andere besteht die Mathematik nur aus Beweisen und Axiomen.
Eine Definition der Mathematik stammt aus dem Jahr 1870 von Charles Sanders Peirce.
\begin{df}
Mathematik ist die Wissenschaft, die notwendige Schlüsse zieht. %\bibitem{qdf1}{Davis/Hersh: Erfahrung Mathematik, Birkhäuser 1985, S. 109}
\cite[s. Literaturverzeichnis]{qdf1}
\label{df1}
\end{df}
Saunders MacLane beschreibt Mathematik wie folgt.\\ \textit{Mathematik bezieht sich auf Beweise und Beweise sind ewig.}
\cite[s. Literaturverzeichnis]{qdf2} % \bibitem{qdf2}{Spektrum der Wissenschaft 1 / 2001, Seite 105;  Spektrum der Wissenschaft Verlagsgesellschaft}
\\ Ein weiterer Aspekt der Mathematik ist, dass man Zahlen, Punkte und Funktionen weder hören, schmecken, sehen noch riechen kann.\\
Auch wenn es Symbole dafür gibt, bleiben die Dinge an sich den menschlichen Sinnen nicht zugänglich.
Das einzige, mit dem die Menschen die Mathematik erfassen können, ist der Geist. Die Mathematik beschäftigt sich also mit Objekten, die nicht in der Natur vorkommen, und ist damit eine Geisteswissenschaft.\\
Die Mathematik fällt Urteile allein durch logische Deduktion (siehe Definition \ref{df1}).
Während z.\,B. physikalische Aussagen von den äußeren Umständen abhängig sind, gilt eine mathematische Aussage unabhängig davon, wer sie wann trifft. Eine bewiesene Aussage ist in der Mathematik unabhängig vom Beweisansatz immer gültig.
Sicherlich dient die Mathematik auch anderen Wissenschaften, jedoch ist sie auch heute noch, wie bereits bei den Griechen, eine eigenständige
Wissenschaft, welche unabhängig von ihren Anwendungen betrieben wird. 
